\usepackage{tikz}
\usetikzlibrary{arrows}
\usetikzlibrary{calc}

%\tikzstyle{vide} = [minimum size=0cm]
\tikzstyle{p1} = [style={circle,fill=blue,draw}, anchor=mid]
\tikzstyle{p2} = [style={circle,fill=red,draw}, anchor=mid]
%\tikzstyle{milieu} = []
%\tikzstyle{fake} = [draw=none]
%\tikzstyle{trait} = []
%\tikzstyle{rempli1} = [draw=none, fill=red!20]
%\tikzstyle{rempli2} = [draw=none, fill=blue!20]

\newcommand*{\hauteur}{1.5}
\newcommand*{\hauteurr}{1}

\newcommand{\drawbulletforplayer}[2]{
  \ifthenelse{\equal{#2}{1}}%
      {\node[p1] at (#1,\hauteur) {};}%
      {\node[p2] at (#1,-\hauteur) {};};
}

\newcommand{\drawrectforplayer}[3]{
  \ifthenelse{\equal{#3}{1}}%
      {\filldraw[fill=blue] (#1, \hauteurr) -- (#2, \hauteurr) -- (#2, 0) -- (#1, 0) -- cycle;}%
      {\filldraw[fill=red] (#1, 0) -- (#2, 0) -- (#2, -\hauteurr) -- (#1, -\hauteurr) -- cycle;};
}

%\newcommand{\drawrect}[5]{
%  \coordinate (p1) at (#1, #3); %(\xmin, \ymin)
%  \coordinate (p2) at (#2, #3); %(\xmax, \ymin)
%  \coordinate (p3) at (#2, #4); %(\xmax, \ymax)
%  \coordinate (p4) at (#1, #4); %(\xmin, \ymaxx)
%  \filldraw[fill=#5] (p1) -- (p2) -- (p3) -- (p4) -- cycle;
%}