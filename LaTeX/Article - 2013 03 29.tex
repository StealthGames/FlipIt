\documentclass[a4paper]{article}

\usepackage[utf8]{inputenc}
\usepackage[T1]{fontenc}
\usepackage[english]{babel}
\usepackage[a4paper]{geometry}
\usepackage{amsmath}


\title{FlipIt}
\author{Simon Forest \and Baptiste Lefebvre \and Vincent Vidal}
\date{\today}

\sloppy

%%%%%%%%%%%%%%%%%%%%%%%%%%%%%%%%%%%%%%%%%%%%%%%%%%%%%%%%%%%%%%%%%%%%%%%%%%%%%%%%%%%%%%%%%%%%%
%                                                                                           %
%%%%%%%%%%%%%%%%%%%%%%%%%%%%%%%%%%%%%%%%%%%%%%%%%%%%%%%%%%%%%%%%%%%%%%%%%%%%%%%%%%%%%%%%%%%%%

\newcommand{\un}{1}
\newcommand{\qquote}[1]{\emph{#1}}
\newcommand{\iinte}[4]{\int_{#1}^{#2} #3 \mathrm{d} #4}            % <- 
\newcommand{\mbb}[1]{\textbf{#1}}                                  % <- mathbb


\newcommand{\pars}[1]{\left(#1\right)}
\newcommand{\manque}[1]{\fbox{\begin{minipage}[t]{1\columnwidth}\textbf{A faire :}\\\texttt{#1}\end{minipage}}}
\newcommand{\segint}[1]{\left[\!\left[#1\right]\!\right]}
\begin{document}

\maketitle

\begin{abstract}
Ideal cryptography systems are based on a secret, a key, however advanced Persistent
Threat (APT) have undermined secure protocols. \emph{FlipIt} [1], [2] is a recent
two-player game between an attacker and defender. It provides a simple and elegant
framework to formalize their interactions and allows the description of pratical
threats.

We propose a variant of FlipIt, a two-player game where players compete to control
a shared ressource. Players can move at any given time, taking control of the
ressource. However the identity of the player controlling the ressource is not
revealed until a player actually moves. We consider that the average move rate by
player is bounded, the number of moves made by player up to an including time can
not be greater than a constant times this time.

First we introduce formal definitions and notation in order to make an exhaustive
description of our variant of FliIt.

\texttt{[Describe here the end of the structure of this article]}

\texttt{[Describe here the results of this article]}
\end{abstract}


\section{Formal Definition and Notation}
This section gives a formal definition of the variant of FlipIt.

\paragraph{Players}
There are two players identified with 0 and 1. Keep the game symetric between the
two players will be useful for our studies.

\paragraph{Time}
The game begins at time $ t = 0 $ and continues indefinitely. It is viewed as being
continuous.

\paragraph{Game state}
$ C \left( t \right) $ denotes the current player controlling the ressource at
time $ t $, $ C \left( t \right) $ is either $ 0 $ or $ 1 $. For $i = 0, 1 $ let
$ C_i \left( t \right) = \un_{ \{ C \left( t \right) = i \} } $ two
\qquote{indicator functions}.

\paragraph{Moves}
A player may \qquote{flip the ressource} or \qquote{move} at any time, but he is
not allowed to exceed a fixed \qquote{average move rate}. A player cannot move  
more than once at a given time. If different players play at the same time then 
the moves \qquote{cancel} and everything goes such as nothing happened. We      
introduce the following notations:
\[ \textbf{t} = t_0 \: t_1 \: \dots \: t_n \: \dots \]
denotes the \qquote{sequence of moves times by both players},
\newpage
\[ \textbf{n} \left( t \right) = \mathrm{Card} \{t_n : t_n \leq t \} \]
denotes the \qquote{number of moves made by both players up to an including time $ t $},
\[ \boldsymbol\alpha \left( t \right) = \frac{ \textbf{n} \left( t \right) }{t} \]
denotes the \qquote{average move rate by both players}. We can define notations 
for same entities only for one player, let consider player $ i $:
\[ \textbf{t}_i = t_{i,0} \: t_{i,1} \: \dots \: t_{i,n} \: \dots
\qquad \textbf{n}_i \left( t \right) = \mathrm{Card} \{t_{i,n} : t_{i,n} \leq t \}
\qquad \boldsymbol\alpha_i \left( t \right) = \frac{ \textbf{n}_i \left( t \right) }{t} \]
denote the \qquote{sequence of moves times by player $ i $}, the \qquote{number 
of moves made by players $ i $ up to an including time $ t $}, denotes the      
\qquote{average move rate by player $ i $}.

\paragraph{Rule}
For a given player i let $ \textbf{A}_i \in \mbb{R}^+ $ denotes the \qquote{average   
move rate upper bound for player $ i $}. The moves of this player have to verify
at any time $ t $:
\[ \boldsymbol\alpha_i \left( t \right) \leq \textbf{A}_i \]
Which means that a player capitalized time into opportunities to play.

\paragraph{Gain}
For a given game at any time $ t $:
\[ \textbf{G}_i \left( t \right) = \iinte{0}{t}{\textbf{C}_i \left( u \right)}{u} \]
denotes the \qquote{total gain by player $ i $ up to time $ t $} which is not really
convenient for our studies. We introduce another metric:
\[ \boldsymbol\gamma_i \left( t \right) = \frac{\textbf{G}_i \left( t \right)}{t} \]
which denotes the \qquote{average gain rate for player $ i $ up to time $ t $} and
leads to the value:
\[ \boldsymbol\Gamma_i = \lim_{ t \rightarrow +\infty} \boldsymbol\gamma_i \left( t \right) \]
This \qquote{average gain rate for player $ i $} will found our game's strategy comparisons.

\section{Playing periodically}
We assume here that both players have a periodic strategy.Let $ \varphi_i $ and 
$ \alpha_i $ the phase and the period of the player $i$. So, we have for each $n$,
\[ t_{i,n} = \varphi_i + n.\alpha_i \]

We can assume that $\alpha_0 \leq \alpha_1 $. 
As the average gain rate doesn't depend of the initial time, we can assume that $ \varphi_0 = 0 $ too.
Let $ \Gamma\left(\alpha_0,\:\alpha_1,\:\varphi_1\right) $ be the average gain rate for the player $ 0 $.
It's easy to sea that :
\[ \Gamma\left(\alpha_0,\:\alpha_1,\:\varphi_1\right) = \Gamma\left(1,\:\frac{\alpha_1}{\alpha_0},\:\varphi_1\right) \]

\texttt{[Superbe dessin]}

%\[ \Gamma\left(1,\:\frac{\alpha_1}{\alpha_0},\:\varphi_1\right)
%= \lim_{N\rightarrow +\infty} \frac{1}{N}
%\underset{n\geq 0}{\underset{\alpha n\leq N}{\bigsum}}\]


\section{Playing exponentially against periodic strategy}
Let's assume here that one player have a periodical stategy,
and that the other one has an exponential strategy.
We will denote :
$ \Gamma \left( \alpha,\:\varphi,\:\lambda \right) $
the average gain rate for the player 0 with :

\[t_{0,\:n}=n.\alpha+\varphi\qquad\text{et}\qquad
\proba{t_{1,\:n+1}\leq t_{1,n}+\delta\;|\;t_{1,\:n+1}\geq t_{1,n}}
  =\lambda.\int_0^\delta \ee{-\lambda.x}.\dd{x}\]

\manque{Démonstration}

We can proove that, for $ \varphi = 0 $ we have :

\[ \Gamma \left( \alpha,\:0,\:\lambda \right) = \int_0^1x.p\left(x\right)\dd{x} \]
with $p\left(x\right) = \lambda.\ee{-\lambda.x} $


\begin{thebibliography}{9}

\bibitem{illegi}
  R. L. Rivest,
  \emph{Illegitimi non carborundum},
  Invited keynote talk given at CRYPTO 2011,
  August 15, 2011.

\bibitem{flipit}
  M.van Dijk, A. Juels, A. Oprea, and R. L. Rivest,
  \emph{FlipIt: The Game of "Stealthy Takeover"},
  To appear in Journal of Cryptology,
  2012.

\bibitem{defend}
  K. D. Bowers, M. van Dijk, R. Griffin, A. Juels, A. Oprea, R. L. Rivest, and N. Triandopoulos,
  \emph{Defending against the Unknown Enemy: Applying FlipIt to System Security},
  GameSec,
  2012.

\end{thebibliography}

\end{document}
