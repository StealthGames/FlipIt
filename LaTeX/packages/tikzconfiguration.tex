\usepackage{tikz}
\usetikzlibrary{arrows}
\usetikzlibrary{calc}

%\tikzstyle{vide} = [minimum size=0cm]
\tikzstyle{p1} = [style={circle,fill=blue,draw}, anchor=mid]
\tikzstyle{p2} = [style={circle,fill=red!70,draw}, anchor=mid]
\tikzstyle{rect1} = [fill=blue!50]
\tikzstyle{rect2} = [fill=red!30]
\tikzstyle{fleche} = [<->, >=stealth', ultra thick]
\tikzstyle{texte} = [font=\bfseries\LARGE]
\tikzstyle{ligne} = [gray]
%\tikzstyle{trait} = []
%\tikzstyle{rempli1} = [draw=none, fill=red!20]
%\tikzstyle{rempli2} = [draw=none, fill=blue!20]

\newcommand*{\hauteur}{1.5}
\newcommand*{\hauteurr}{1}

\newcommand{\drawbulletforplayer}[2]{
  \ifthenelse{\equal{#2}{1}}%
      {\node[p1] at (#1,\hauteur) {};}%
      {\node[p2] at (#1,-\hauteur) {};};
}
\newcommand{\drawbulletforplayerWithText}[3]{
  \ifthenelse{\equal{#2}{1}}%
      {\node[p1] at (#1,\hauteur) {};
       \node[] at (#1,\hauteur + 0.5) {#3};}%
      {\node[p2] at (#1,-\hauteur) {};
       \node[] at (#1,-\hauteur - 0.5) {#3};};
}

\newcommand{\drawrectforplayer}[3]{
  \ifthenelse{\equal{#3}{1}}%
      {\filldraw[rect1] (#1, \hauteurr) -- (#2, \hauteurr) -- (#2, 0) -- (#1, 0) -- cycle;}%
      {\filldraw[rect2] (#1, 0) -- (#2, 0) -- (#2, -\hauteurr) -- (#1, -\hauteurr) -- cycle;};
}
\newcommand{\drawtextforplayer}[3]{
  \ifthenelse{\equal{#2}{1}}%
      {\node[texte] at (#1, \hauteur+.5) {#3}}%
      {\node[texte] at (#1, -\hauteur-.5) {#3}};
}
\newcommand{\drawarrowforplayerwithtext}[4]{
  \ifthenelse{\equal{#3}{1}}%
      {\draw[fleche] (#1, \hauteurr*.5) -- node[texte, above, pos=.5]{#4} (#2, \hauteurr*.5)}%
      {\draw[fleche] (#1, -\hauteurr*.5) -- node[texte, below, pos=.5]{#4} (#2, -\hauteurr*.5)};
}
\newcommand{\drawverticalline}[1]{
  \draw[ligne] (#1, -\hauteur) -- (#1, +\hauteur+1.);
}
\newcommand{\drawrectforplayerWithText}[4]{
  \ifthenelse{\equal{#3}{1}}%
      {\filldraw[rect1] (#1, \hauteurr) -- (#2, \hauteurr) -- (#2, 0) -- (#1, 0) -- cycle;
       \node[] at (#1*0.5+#2*0.5, \hauteurr*0.5) {#4};}%
      {\filldraw[rect2] (#1, 0) -- (#2, 0) -- (#2, -\hauteurr) -- (#1, -\hauteurr) -- cycle;
       \node[] at (#1*0.5+#2*0.5, -\hauteurr*0.5) {#4};};
}

%      \node[] at ((#1+#2)*0.5, -\hauteurr*0.5) {#4};
%\newcommand{\drawrect}[5]{
%  \coordinate (p1) at (#1, #3); %(\xmin, \ymin)
%  \coordinate (p2) at (#2, #3); %(\xmax, \ymin)
%  \coordinate (p3) at (#2, #4); %(\xmax, \ymax)
%  \coordinate (p4) at (#1, #4); %(\xmin, \ymaxx)
%  \filldraw[fill=#5] (p1) -- (p2) -- (p3) -- (p4) -- cycle;
%}