\documentclass[a4paper]{article}

\usepackage{tikz}
\usetikzlibrary{arrows}
\usetikzlibrary{calc}

\usepackage{rotating}

\usepackage[utf8]{inputenc}
\usepackage[T1]{fontenc}
\usepackage[francais]{babel} 
\usepackage[textwidth=16cm,textheight=24cm]{geometry}

\tikzstyle{vide} = [minimum size=0cm]
\tikzstyle{p1} = [style={circle,fill=red,draw}, anchor=mid]
\tikzstyle{pf1} = [style={circle,fill=red,draw}, right=0cm, above=0cm, anchor=mid]
\tikzstyle{p2} = [style={circle,fill=blue,draw}, anchor=mid]
\tikzstyle{pf2} = [style={circle,fill=blue,draw}, pos=0.5, right=0cm, above=0cm, anchor=mid]
\tikzstyle{milieu} = []
\tikzstyle{fake} = [draw=none]
\tikzstyle{trait} = []
\tikzstyle{rempli1} = [draw=none, fill=red!20]
\tikzstyle{rempli2} = [draw=none, fill=blue!20]


\begin{document}

\begin{figure}[!t, scale=0.8]
\begin{tikzpicture}[>=stealth',shorten >=1pt,auto,node distance=1cm,thick]

	\node[] (centre) {};	
	\node[vide, node distance = 8cm] (gauchem)[left of=centre] {};
	\node[vide, node distance = 8cm] (droitem)[right of=centre] {};
	\node[vide, node distance = 1cm] (gaucheh)[above of=gauchem] {};
	\node[vide, node distance = 1cm] (droiteh)[above of=droitem] {};
	\node[vide, node distance = 1cm] (gaucheb)[below of=gauchem] {};
	\node[vide, node distance = 1cm] (droiteb)[below of=droitem] {};

	\draw[fake] (gaucheh) --%
			node[pf1, pos=0.0](p1x1){}%
			node[pf1, pos=0.2](p1x2){}%
			node[pf1, pos=0.4](p1x3){}%
			node[pf1, pos=0.6](p1x4){}%
			node[pf1, pos=0.8](p1x5){}%
			node[pf1, pos=1.0](p1x6){}%
							 (droiteh);
							 
	\draw[fake] (gaucheb) -- %
			node[pf2, pos=0.07](p2x1){}%
			node[pf2, pos=0.37](p2x2){}%
			node[pf2, pos=0.67](p2x3){}%
			node[pf2, pos=0.97](p2x4){}%
							(droiteb);
	
	\draw[rempli1] (p1x1) -| ($(p2x1.mid)+(0,1)$) -| (p1x1);
	\draw[rempli1] (p1x2) -| ($(p2x2.mid)+(0,1)$) -| (p1x2);
	\draw[rempli1] (p1x3) -| ($(p2x3.mid)+(0,1)$) -| (p1x3);
	\draw[rempli1] (p1x5) -| ($(p2x4.mid)+(0,1)$) -| (p1x5);
	
	\draw[rempli2] (p2x1) -| ($(p1x2.mid)+(0,-1)$) -| (p2x1);
	\draw[rempli2] (p2x2) -| ($(p1x3.mid)+(0,-1)$) -| (p2x2);
	\draw[rempli2] (p2x3) -| ($(p1x5.mid)+(0,-1)$) -| (p2x3);
	\draw[rempli2] (p2x4) -| ($(p1x6.mid)+(0,-1)$) -| (p2x4);
	
	\draw[milieu] (gauchem) -- (droitem);
	
	\draw[trait] (p1x1) -| (p2x1) -| (p1x2) -| (p2x2) -| (p1x3) -- (p1x4) -| (p2x3) -| (p1x5) -| (p2x4) -| (p1x6);

\end{tikzpicture}
\end{figure}



\end{document}