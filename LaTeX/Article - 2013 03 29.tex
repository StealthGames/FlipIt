\documentclass[a4paper]{article}

\usepackage[utf8]{inputenc}
\usepackage[T1]{fontenc}
\usepackage[english]{babel}
\usepackage[a4paper]{geometry}


\title{FlipIt}
\author{Simon Forest \and Baptiste Lefebvre \and Vincent Vidal}
\date{\today}

\sloppy


\begin{document}

\maketitle

\begin{abstract}
Ideal cryptography systems are based on a secret, a key, however advanced Persistent
Threat (APT) have undermined secure protocols. \emph{FlipIt} [1], [2] is a recent
two-player game between an attacker and defender. It provides a simple and elegant
framework to formalize their interactions and allows the description of pratical
threats.

We propose a variant of FlipIt, a two-player game where players compete to control
a shared ressource. Players can move at any given time, taking control of the
ressource. However the identity of the player controlling the ressource is not
revealed until a player actually moves. We consider that the average move rate by
player is bounded, the number of moves made by player up to an including time can
not be greater than a constant times this time.

\#\#\# Describe here the structure of this article. \#\#\#

\#\#\# Describe here the results of this article. \#\#\#
\end{abstract}

\section{Formal Definition and Notation}
This section gives a formal definition of the variant of FlipIt.

\paragraph{Players}
There are two players identified with 0 and 1. Keep the game symetric between the
two players will be useful for our studies.

\paragraph{Time}
The game begins at time \(t = 0\) and continues indefinitely. It is viewed as being
continuous.

\paragraph{Game state}

\section{Playing periodically}
\#\#\# Describe here the playing periodically study. \#\#\#


\begin{thebibliography}{9}

\bibitem{illegi}
  R. L. Rivest,
  \emph{Illegitimi non carborundum},
  Invited keynote talk given at CRYPTO 2011,
  August 15, 2011.

\bibitem{flipit}
  M.van Dijk, A. Juels, A. Oprea, and R. L. Rivest,
  \emph{FlipIt: The Game of "Stealthy Takeover"},
  To appear in Journal of Cryptology,
  2012.

\bibitem{defend}
  K. D. Bowers, M. van Dijk, R. Griffin, A. Juels, A. Oprea, R. L. Rivest, and N. Triandopoulos,
  \emph{Defending against the Unknown Enemy: Applying FlipIt to System Security},
  GameSec,
  2012.

\end{thebibliography}

\end{document}
